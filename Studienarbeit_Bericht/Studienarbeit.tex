\documentclass[a4paper,11pt]{article}

\usepackage[T1]{fontenc}
\usepackage[utf8]{inputenc}
\usepackage{graphicx}
\usepackage{xcolor}
\usepackage[T1]{fontenc}
\renewcommand\familydefault{\sfdefault}
\usepackage{tgheros}
\usepackage[defaultmono]{droidmono}
\usepackage[export]{adjustbox}
\usepackage{multirow}
\usepackage{tabularx}
\usepackage{float}
\usepackage{wrapfig}
\usepackage{amsmath,amssymb,amsthm,textcomp}
\usepackage{enumerate}
\usepackage{multicol}
\usepackage{tikz}
\usepackage[ngerman]{babel}
\usepackage{xpatch}
\usepackage[bottom]{footmisc}
\usepackage{footnote}
\usepackage{acronym}

\bibliographystyle{unnsert}
\usepackage{geometry}
\geometry{total={210mm,297mm},
left=25mm,right=25mm,%
bindingoffset=0mm, top=20mm,bottom=20mm}


%\linespread{1.3}



% custom theorems if needed
\newtheoremstyle{mytheor}
    {1ex}{1ex}{\normalfont}{0pt}{\scshape}{.}{1ex}
    {{\thmname{#1 }}{\thmnumber{#2}}{\thmnote{ (#3)}}}

\theoremstyle{mytheor}
\newtheorem{defi}{Definition}

% my own titles
\makeatletter
\renewcommand{\maketitle}{
\begin{center}
\vspace{2ex}
{\huge \textsc{\@title}}
\vspace{1ex}
\\
\linia\\
\@author \hfill \@date
\vspace{4ex}
\end{center}
}
\makeatother
%%%

% custom footers and headers
\usepackage{fancyhdr}
\pagestyle{fancy}
\lhead{}
\chead{}
\rhead{}
\cfoot{}
\rfoot{Page \thepage}
\renewcommand{\headrulewidth}{0pt}
\renewcommand{\footrulewidth}{0pt}%%%----------%%%----------%%%----------%%%----------%%%

\begin{document}
\begin{titlepage}
	\begin{figure}
	\includegraphics[width=0.3\textwidth,right]{{logo.jpg}}
	\end{figure}
	\centering
	{\scshape\LARGE Hochschule Esslingen\par}
	\vspace{1cm}
	{\scshape\Large Studienarbeit\par}
	\vspace{1.5cm}
	{\huge\bfseries Sirenen Erkennung  durch ein Neuronales Netz\par}
	\vspace{2cm}
	{\Large\itshape Seifeddine Mhiri\par}
	\vfill
{\Large\itshape im Studiengang Technische Informatik\\
der Fakultät Informationstechnik\\
Sommersemester 2020\par}
\vspace{5cm}
{\Large\itshape ------------------------------------------------------------------- \\ 

Betreuer: 
Prof. Dr.-Ing. Thao Dang \\
  -------------------------------------------------------------------}
\end{titlepage}
\newpage

\vspace*{150px}
\noindent\huge{Eidesstattliche Erklärung}\\\\
\large{\\
Hiermit erkläre ich, die vorliegende Arbeit selbstständig und unter ausschließlicher Verwendung der angegebenen Literatur und Hilfsmittel erstellt zu haben.\\\\
Die Arbeit wurde bisher in gleicher oder ähnlicher Form keiner andern Prüfungsbehörde vorgelegt und auch nicht veröffentlicht.\\
\\Esslingen, den 1.März.2020 \hspace{2.5cm}{\_\_\_\_\_\_\_\_\_\_\_\_\_\_\_\_\_}
\\\hspace*{9.3cm}Unterschrift\\}
\vspace*{250px}
\newpage
\vspace*{150px}
\noindent\huge{Vorwort}\\\\
\large{Einführung: Den Bachelor-StudentInnen  für den Studiengang Technische Informatik der Fakultät Informationstechnik an der Hochschule Esslingen  ist es Pflicht  im sechsten Semester ihres Hauptfachstudiums das Modul „Studienprojekt“  zu belegen. Ich habe mich  dazu  entschieden,  im  Rahmen  eines  Studienprojektes  meine  im  Studium  erworbenen Kompetenzen   zu   erweitern   und   zu   vertiefen.   Für   die Erarbeitung   eines   möglichen Projektverlaufes  habe  ich  hierzu  neben  den  über  das  Institut  bereitgestellten  Informationen mehrere  Beratungsgespräche  mit  meinem  fachinternen Betreuer  wahrgenommen.  Da  ich mich  sehr  für  Maschinelles Lernen  interessiere und zum Zeitpunkt der Ideenfindung gerade ein Seminare über " künstliche Intelligenz und Deep lerning" \space belegt habe, ordnete  ich  mein  Studienprojekt  grob  in  das  Thema  Sirenen Erkennung durch ein Neuronalesnetz ein. }
\newpage
\tableofcontents
\newpage
\addcontentsline{toc}{section}{Abbildungsverzeichnis}

\listoffigures
\newpage
\addcontentsline{toc}{section}{Abkürzungsverzeichnis}
\noindent\huge{Abkürzungsverzeichnis}
\\ \large
\begin{acronym}[Bash]
\acro{HAL}{Hardware Abstraction Layer}
\end{acronym}
\section{Problemstellung}
\subsection{Einführeung}
\subsection{Serinen}
\subsection{Deep Leaning}
\section{Grundlagen}
\subsection{Python}
\subsection{Jupyter-notebook}
hey
\subsection{Librosa}
\subsection{Keras}
\section{Impilimentireung}
\subsection{Step 1: Data Exploration and Visualisation}
\subsection{Step 2: Data Preprocessing and Data Splitting}
\subsection{Step 3: Model Training and Evaluation}
\subsection{Step 4: Testing}
hello

\section{Auswertung}
\addcontentsline{toc}{section}{Literaturverzeichnis}
\documentclass[a4paper,11pt]{article}

\usepackage[T1]{fontenc}
\usepackage[utf8]{inputenc}
\usepackage{graphicx}
\usepackage{xcolor}
\usepackage[T1]{fontenc}
\renewcommand\familydefault{\sfdefault}
\usepackage{tgheros}
\usepackage[defaultmono]{droidmono}
\usepackage[export]{adjustbox}
\usepackage{multirow}
\usepackage{tabularx}
\usepackage{float}
\usepackage{wrapfig}
\usepackage{amsmath,amssymb,amsthm,textcomp}
\usepackage{enumerate}
\usepackage{multicol}
\usepackage{tikz}
\usepackage[ngerman]{babel}
\usepackage{xpatch}
\usepackage[bottom]{footmisc}
\usepackage{footnote}
\usepackage{acronym}

\bibliographystyle{unnsert}
\usepackage{geometry}
\geometry{total={210mm,297mm},
left=25mm,right=25mm,%
bindingoffset=0mm, top=20mm,bottom=20mm}


%\linespread{1.3}



% custom theorems if needed
\newtheoremstyle{mytheor}
    {1ex}{1ex}{\normalfont}{0pt}{\scshape}{.}{1ex}
    {{\thmname{#1 }}{\thmnumber{#2}}{\thmnote{ (#3)}}}

\theoremstyle{mytheor}
\newtheorem{defi}{Definition}

% my own titles
\makeatletter
\renewcommand{\maketitle}{
\begin{center}
\vspace{2ex}
{\huge \textsc{\@title}}
\vspace{1ex}
\\
\linia\\
\@author \hfill \@date
\vspace{4ex}
\end{center}
}
\makeatother
%%%

% custom footers and headers
\usepackage{fancyhdr}
\pagestyle{fancy}
\lhead{}
\chead{}
\rhead{}
\cfoot{}
\rfoot{Page \thepage}
\renewcommand{\headrulewidth}{0pt}
\renewcommand{\footrulewidth}{0pt}%%%----------%%%----------%%%----------%%%----------%%%

\begin{document}
\begin{titlepage}
	\begin{figure}
	\includegraphics[width=0.3\textwidth,right]{{logo.jpg}}
	\end{figure}
	\centering
	{\scshape\LARGE Hochschule Esslingen\par}
	\vspace{1cm}
	{\scshape\Large Studienarbeit\par}
	\vspace{1.5cm}
	{\huge\bfseries Sirenen Erkennung  durch ein Neuronales Netz\par}
	\vspace{2cm}
	{\Large\itshape Seifeddine Mhiri\par}
	\vfill
{\Large\itshape im Studiengang Technische Informatik\\
der Fakultät Informationstechnik\\
Sommersemester 2020\par}
\vspace{5cm}
{\Large\itshape ------------------------------------------------------------------- \\ 

Betreuer: 
Prof. Dr.-Ing. Thao Dang \\
  -------------------------------------------------------------------}
\end{titlepage}
\newpage

\vspace*{150px}
\noindent\huge{Eidesstattliche Erklärung}\\\\
\large{\\
Hiermit erkläre ich, die vorliegende Arbeit selbstständig und unter ausschließlicher Verwendung der angegebenen Literatur und Hilfsmittel erstellt zu haben.\\\\
Die Arbeit wurde bisher in gleicher oder ähnlicher Form keiner andern Prüfungsbehörde vorgelegt und auch nicht veröffentlicht.\\
\\Esslingen, den 1.März.2020 \hspace{2.5cm}{\_\_\_\_\_\_\_\_\_\_\_\_\_\_\_\_\_}
\\\hspace*{9.3cm}Unterschrift\\}
\vspace*{250px}
\newpage
\vspace*{150px}
\noindent\huge{Vorwort}\\\\
\large{Einführung: Den Bachelor-StudentInnen  für den Studiengang Technische Informatik der Fakultät Informationstechnik an der Hochschule Esslingen  ist es Pflicht  im sechsten Semester ihres Hauptfachstudiums das Modul „Studienprojekt“  zu belegen. Ich habe mich  dazu  entschieden,  im  Rahmen  eines  Studienprojektes  meine  im  Studium  erworbenen Kompetenzen   zu   erweitern   und   zu   vertiefen.   Für   die Erarbeitung   eines   möglichen Projektverlaufes  habe  ich  hierzu  neben  den  über  das  Institut  bereitgestellten  Informationen mehrere  Beratungsgespräche  mit  meinem  fachinternen Betreuer  wahrgenommen.  Da  ich mich  sehr  für  Maschinelles Lernen  interessiere und zum Zeitpunkt der Ideenfindung gerade ein Seminare über " künstliche Intelligenz und Deep lerning" \space belegt habe, ordnete  ich  mein  Studienprojekt  grob  in  das  Thema  Sirenen Erkennung durch ein Neuronalesnetz ein. }
\newpage
\tableofcontents
\newpage
\addcontentsline{toc}{section}{Abbildungsverzeichnis}

\listoffigures
\newpage
\addcontentsline{toc}{section}{Abkürzungsverzeichnis}
\noindent\huge{Abkürzungsverzeichnis}
\\ \large
\begin{acronym}[Bash]
\acro{HAL}{Hardware Abstraction Layer}
\end{acronym}
\section{Einführeung}
\subsection{Serinen}
\subsection{Deep Leaning}
\section{Grundlagen}
\subsection{Python}
\subsection{Jupyter-notebook}
hey
\subsection{Librosa}
\subsection{Keras}
\section{Impilimentireung}
\subsection{Step 1: Data Exploration and Visualisation}
\subsection{Step 2: Data Preprocessing and Data Splitting}
\subsection{Step 3: Model Training and Evaluation}
\subsection{Step 4: Testing}
  hhhhhhhhhhhhhhhhhhhhhhhhhhhhhh

\section{Auswertung}
\addcontentsline{toc}{section}{Literaturverzeichnis}
\bibliography{literatur}
\begin{thebibliography}{9}

\end{thebibliography}
\end{document}	

\end{document}	